% arara: pdflatex: { synctex: yes }
% arara: makeindex: { style: ctuthesis }
% arara: bibtex

% The class takes all the key=value arguments that \ctusetup does,
% and a couple more: draft and oneside
\documentclass[twoside]{ctuthesis}

\ctusetup{
	preprint = \ctuverlog,
	mainlanguage = english,
%	titlelanguage = czech,
%	mainlanguage = czech,
	otherlanguages = {slovak, czech},
	title-czech = {Lokalizace a segmentace in-vivo ultrazvukových obrazů karotidy},
	title-english = {Localization and segmentation of in-vivo ultrasound carotid artery images},
%	subtitle-czech = {Cesta do tajů kdovíčeho},
%	subtitle-english = {Journey to the who-knows-what wondeland},
	doctype = M,
	faculty = F3,
	department-czech = {Katedra počítačů},
	department-english = {Department of Computer Science},
	author = {Martin Kostelanský},
	supervisor = {prof. Dr. Ing. Jan Kybic},
	supervisor-address = {Center for Machine Perception, \\ Karlovo náměstí 13, \\ Prague 2},
%	supervisor-specialist = {John Doe},
	fieldofstudy-english = {Open Informatics},
	subfieldofstudy-english = {Artificial Intelligence},
	fieldofstudy-czech = {Otevřená informatika},
	subfieldofstudy-czech = {Umělá inteligence},
	keywords-czech = {lekarsky ultrazvuk, stenoza karotid, konvolucna neuronova siet},
	keywords-english = {medcial ultrasound, carotid stenosis, convolutional neural network},
	day = 5,
	month = 1,
	year = 2021,
	specification-file = {zadanie.pdf},
%	front-specification = true,
%	front-list-of-figures = false,
%	front-list-of-tables = false,
%	monochrome = true,
%	layout-short = true,
}

\ctuprocess

\addto\ctucaptionsczech{%
	\def\supervisorname{Vedoucí}%
	\def\subfieldofstudyname{Studijní program}%
}

\ctutemplateset{maketitle twocolumn default}{
	\begin{twocolumnfrontmatterpage}
		\ctutemplate{twocolumn.thanks}
		\ctutemplate{twocolumn.declaration}
		\ctutemplate{twocolumn.abstract.in.titlelanguage}
		\ctutemplate{twocolumn.abstract.in.secondlanguage}
		\ctutemplate{twocolumn.tableofcontents}
		\ctutemplate{twocolumn.listoffigures}
	\end{twocolumnfrontmatterpage}
}

% Theorem declarations, this is the reasonable default, anybody can do what they wish.
% If you prefer theorems in italics rather than slanted, use \theoremstyle{plainit}
\theoremstyle{plain}
\newtheorem{theorem}{Theorem}[chapter]
\newtheorem{corollary}[theorem]{Corollary}
\newtheorem{lemma}[theorem]{Lemma}
\newtheorem{proposition}[theorem]{Proposition}

\theoremstyle{definition}
\newtheorem{definition}[theorem]{Definition}
\newtheorem{example}[theorem]{Example}
\newtheorem{conjecture}[theorem]{Conjecture}

\theoremstyle{note}
\newtheorem*{remark*}{Remark}
\newtheorem{remark}[theorem]{Remark}

\setlength{\parskip}{5ex plus 0.2ex minus 0.2ex}

% Abstract in English
\begin{abstract-english}
	To write at the end.
		
\end{abstract-english}
	
% Abstract in Czech
\begin{abstract-czech}
To write at the end.

\end{abstract-czech}


% Acknowledgements / Podekovani
\begin{thanks}
I would like to thank to professor Kybic for his valuable counsel and answering my e-mails after 10 p.m.
Dedicated to my family and friends for their endless support during my studies.
\end{thanks}

% Declaration / Prohlaseni
\begin{declaration}
I declare that I have worked on this thesis independently using only the sources listed in the bibliography.

In Prague, \ctufield{day}.~\monthinlanguage{title}~\ctufield{year}
\end{declaration}

% Only for testing purposes
\listfiles
\usepackage[pagewise]{lineno}
\usepackage{lipsum,blindtext}
\usepackage{mathrsfs} % provides \mathscr used in the ridiculous examples

\begin{document}

\maketitle

%%%%%%%%%%%%%%%%%%%%%%%%%%%%%%%%%%%%%%%%%%%%%%%%%%%%%%%%%%%%%%%%%%%%%%%%%%%%%%%%

\chapter{Introduction}
Artificial inteligence is a science field which aims to build intelligent systems
and understand the principles behind it \cite{AIMAbook}. Most of the researches
assumes that ability to learn is a predisposition for intelligence. \cite{KONONENKO200189}
Machine learning is a subfield of AI, which focuses on learning a behavior from data.
The rise of deep learning, an area of machine learning which focuses on deep neural networks,
started around 2010, when this models outperformed algorithm based methods \cite{DL_BOOK}. 
Since than, deep learning has been applied in a wide range of applications, 
from natural language processing \cite{mikolov2013efficient}, finance \cite{Ding2015DeepLF}, 
image processing \cite{AlexNet} or medical diagnosis \cite{6868045}.

Paragraph - Image processing in medicine

Paragraph - Thesis - direct topis intro

%%%%%%%%%%%%%%%%%%%%%%%%%%%%%%%%%%%%%%%%%%%%%%%%%%%%%%%%%%%%%%%%%%%%%%%%%%%%%%%%


%!TEX ROOT=ctutest.tex

\chapter{Goals and objectives}
The aim of this work is to create a collection of machine learning models, which
will be used in both, further research and image processing pipeline. The first
goal is to create a model which would be able to categorize different types of
ultrasound images, namely transversal, longitudinal and doppler ones. Further, 
we need to localize the common carotid artery (respectively internal carotid artery)
on the images of the first two named categories. As a last step, we need to
design a segmentation model which would segment three particular parts of
an artery with stenosis - vessel, lumen and plaque. Since all of these steps
in processing are interconnected and the outputs of one model are fed into a 
next one, all of the proposed methods need to achieve sufficient performance.


%%%%%%%%%%%%%%%%%%%%%%%%%%%%%%%%%%%%%%%%%%%%%%%%%%%%%%%%%%%%%%%%%%%%%%%%%%%%%%%%


\chapter{Background}

\section{Carotid Artery Stenosis}
To write.

\section{Medical ultrasound}
In general, medical ultrasound is a technique using sound waves in range etc.


%%%%%%%%%%%%%%%%%%%%%%%%%%%%%%%%%%%%%%%%%%%%%%%%%%%%%%%%%%%%%%%%%%%%%%%%%%%%%%%%

\chapter{Existing methods}

\section{Image Classification}
Image classification is one of the primary tasks in the field od image processing.
Its goal is to assing to an image one of predefined categories.
The neural network have achieved breakthrough in this field, namely ones using
convolutional layers. Later, as in many other domains, the deep learning has 
become state of the art in this field. One of the benchmark for this task is ImageNet 
Large Scale Visual Recognition Challenge \cite{russakovsky2015imagenet}, which has begun in 2010.
The task is to crate a model able to classify over 1.4 million images into one thousand 
categories. The size of the annotated dataset with the reduction of training time achieved by using GPU led
to deep architectures \cite{AlexNet}.

After inital successes of deep convolutional neural networks, they have been widely used
applied in many fields, inculuding medical and biology image processing. For example to 
predict breast cancer based on histopathological images \cite{BreastCNN}, to classify
lung pattern for interstitial lung diseases \cite{LungCNN} or to detect and classify
abnormalities on frontal chest radiographs \cite{ChestCNN}.


\subsection{VGG}
Very deep convolutional neural networks for large-scale image recognition \cite{VGG} were 
introduced in 2014, and it achieved both, first and second class in the Classification tracks of 
ImageNet Challenge \cite{IN2014} in the same year.


\subsubsection{Architecture}
Six architectures were introduced in the experiments, each containing six blocks of convolutional
layers separated by max-pooling ones. In the convolutional layers were used filters with
with size 3x3 (in one experiment were used filters with size 1x1 at the end of three convolutional blocks).
With the 3x3 convolutional filters is used stride 1 with zero padding, so the spatial dimensionality
is preserved trough whole block. In the max-pooling layer, a receptor field with size 2x2 is used and with stride 1 it 
reduces the dimensionality to half. Finally, there are three fully-connected layers. First two with 4096 neurons and
the last one with 1000 neuron followed by sigmoid activation function. The two best performing models with 16, respectively 19
layers are described by Figure TODO.


%%%%%%%%%%%%%%%%%%%%%%%%%%%%%%%%%%%%%%%%%%%%%%%%%%%%%%%%%%%%%%%%%%%%%%%%%%%%%%%%


\subsection{ResNet}
The deep neural networks are generally harder to train \cite{HardDnn}. In 2015 was proposed ResNet\cite{ResNet}, 
a deep residual convolutional network. The depth of the networks was pushed even further, up to 152 layers. 
This combination resulted in the first place in the Categorization track of ImageNet Challenge 2015\cite{IN2015}
(ResNet models can be found under MSRA team name).

\subsubsection{Architecture}
The architectures of ResNet followes principles introduced in VGG, and uses mostly convolutional
layers with 3x3 filters, in some versions combined with 1x1 filters. ResNet takes an input of
size 224x224 pixels, which can be translated into 224x224x3 matrix. This input is than processed
by convolutional layer with filter size 7x7 and stride 2, which results in reduction of the 
dimension by two - to 112x112. The output of first layer is fed into only max-pooling layer,
with receptor filed 3x3 and stride 2. The following convolutional part is composed from four
blocks of convolutional layers, which structure varies by the specific network's version.
The dimensionality between convolutional blocks is reduced by increasing stride to two
in the first convolutional layer of a block, instead of using max-pooling, which is used in VGG.
Nerwork contains only on fully-connected layer, which is at the end and is followed by 
soft-max axtivation function, which translates the neorons' output to probabilities of the
one thousand categories.


\subsubsubsection{Residual learning}
The layes trough the neworks are not only connected with the previous ones, but 
they are connected with the residual connections as well. These shourtcuts help to
train such deep network and are based on the assumtion, that a network with these
connections should be able to fit the data as well as the shallower network without them.
TODO add images and describe deeply, add equations. 



%%%%%%%%%%%%%%%%%%%%%%%%%%%%%%%%%%%%%%%%%%%%%%%%%%%%%%%%%%%%%%%%%%%%%%%%%%%%%%%%


\section{Object Localization}
Object localization can be defined as a task to localize objects on an image. Usually
by surounding each one with a rectange TODO fig lara.

\section{From R-CNN to Faster R-CNN}
To write.


\subsection{Faster R-CNN}
To write.

\subsubsection{Architecture}
To write.


%%%%%%%%%%%%%%%%%%%%%%%%%%%%%%%%%%%%%%%%%%%%%%%%%%%%%%%%%%%%%%%%%%%%%%%%%%%%%%%%


\section{Segmentation}
To write.

\subsection{U-net}
To write.

\subsubsection{Architecture}
To write.

\subsubsection{Applications}
To write.

%%%%%%%%%%%%%%%%%%%%%%%%%%%%%%%%%%%%%%%%%%%%%%%%%%%%%%%%%%%%%%%%%%%%%%%%%%%%%%%%


%%%%%%%%%%%%%%%%%%%%%%%%%%%%%%%%%%%%%%%%%%%%%%%%%%%%%%%%%%%%%%%%%%%%%%%%%%%%%%%%


%!TEX ROOT=ctutest.tex

%%%%%%%%%%%%%%%%%%%%%%%%%%%%%%%%%%%%%%%%%%%%%%%%%%%%%%%%%%%%%%%%%%%%%%%%%%%%%%%%

\chapter{Data}
\section{Prague dataset}
To write.

\subsection{Data annotations}
To write.


\section{Brno dataset}
To write.


%%%%%%%%%%%%%%%%%%%%%%%%%%%%%%%%%%%%%%%%%%%%%%%%%%%%%%%%%%%%%%%%%%%%%%%%%%%%%%%%

\chapter{Classification}
Since our dataset contains a patient's images from one examination, to process
them further, we need to categorize the ultrasound images into three main 
categories - longitudinal, traversal and conical #TODO-NAMING.

\section{Dataset}
To write.

\section{Model}
To write.

\section{Experiments and results}
To write.


%%%%%%%%%%%%%%%%%%%%%%%%%%%%%%%%%%%%%%%%%%%%%%%%%%%%%%%%%%%%%%%%%%%%%%%%%%%%%%%%

\chapter{Localization}
The are scanned by an ultrasaound is too big, compared to the our region of interest, 
the carotid artery.


\section{Dataset}
To write.

\section{Model}
To write.

\section{Experiments and results}
To write.


%%%%%%%%%%%%%%%%%%%%%%%%%%%%%%%%%%%%%%%%%%%%%%%%%%%%%%%%%%%%%%%%%%%%%%%%%%%%%%%%

\chapter{Segmentation}

\section{Dataset}
To write.

\section{Model}
To write.

\section{Experiments and results}
To write.


%%%%%%%%%%%%%%%%%%%%%%%%%%%%%%%%%%%%%%%%%%%%%%%%%%%%%%%%%%%%%%%%%%%%%%%%%%%%%%%%

%%%%%%%%%%%%%%%%%%%%%%%%%%%%%%%%%%%%%%%%%%%%%%%%%%%%%%%%%%%%%%%%%%%%%%%%%%%%%%%%

\chapter{Conclusion and Results}
To write.

%%%%%%%%%%%%%%%%%%%%%%%%%%%%%%%%%%%%%%%%%%%%%%%%%%%%%%%%%%%%%%%%%%%%%%%%%%%%%%%%

\appendix

\chapter{Convolutional neural net}
Convolutional neural nets are a family of neural nets, which use a convolutional layer.
Usually they take as an input grid structured data \cite{DL_BOOK}, typicaly images, for example
 one may see as 3D tensor, one dimension for each primary color in RGB encoding. But from their
 first pracical use in reading check system \cite{CHECKS}, they have been applied in many 
 domains, not only necessarily image processing. They have been used in text processing, for
 example in anlayzing sentiment of a text \cite{SENTIMENT_CNN}, time series classification 
\cite{TS_CNN} or in the field of recommender systems \cite{REC_CNN}. Its main component is a
convolutional layer, which is often combined a pooling layer. One may find convolutional 
networks combined with fully-connected or even LSTM layers \cite{LSTM_CNN}. In this section
will be discussed the most used ones - convollutional, pooling and fully connected.


\section{Convolutional layer}
The cornerstone of each convolutional neural net is a convolutional layer. In this layer, 
one or multiple convolutional filters are applied on the layer's input. In the figure TODO we 
can see different convolutional filters applied on an image. In the neural network, each of the 
trainable filters is relatively small and serves as a feature extractor. At each position, 
filter's kernel takes an input from its receptors field, computes it's weighted sum, adds bias 
nd transform the input by a non-linar activation function. This is transformation described by 
the equation TODO add eq. The kernel is continuosly applied over an input, and the ditance
betweed two such operations is called stride. Since convolution reduce the dimension of an
image, the input is usully padded with some constant value (for example 0), to perserve it.
On the image TODO, it is shown convloution on an image of size 5x5, and the size of the 
kernel is 3x3. To keep the spatial dimension, zero padding is used.

\begin{equation} 
	Y_{i,j}=\sigma(b+\sum_{k=0}^{x}\sum_{l=0}^{y}w_{k,l}a_{i+k,j+l}),
\end{equation} 

\section{Pooling layer}
Pooling layers often follows the convolutional ones, and their purpose is to reduce 
the dimensionaluty. Fistly was used average pooling, which computed average value of
the receptor field, shown in equation A.3. In the last years was introduced, max-pooling, 
which propagates the maximum value at each postion \cite{CNNREW}. For example, using pooling
layer with receptor field of size 2x2 and stride 2, reduces dimension to falf. Such example
can be seen on fitgure TODO.

\begin{equation} 
	Y_{i,j}=\frac{1}{xy}\sum_{k=0}^{x}\sum_{l=0}^{y}w_{k,l}a_{i+k,j+l},
\end{equation} 

\begin{equation} 
	Y_{i,j} = \max_{(p,q)\in{\Re_{i,j}}}(x_{p,q}),
\end{equation} 


\section{Fully-connected layer}
Fully-connected layer is composed from one, or multiple neurons, where each neuron is connected
to every unit in previous layer by trainable weight. With added bias and transformation with
non-linear function, the output is forwarded to the next layer.

\begin{equation} 
	Y_{i,j}=\sigma(b +\sum_{k=0}^{n}w_{k,j}a_{k,j-1}),
\end{equation}


\section{Architecture}
Typicaly, input to the CNN is fixed, so the image needs to preprocessed accordingly.
Fistly, an input is proessed by series of convolutional layers grouped in blocks, 
where everu convolutional layer has the same same setting (number of filters, kernel size, 
stride,padding). The dimension between the block is reduced by a pooling layer, and the number
of filters in next conv. block increased. The last part is composed by one or multiple
fully-connected layers. The nubmer of neurons is depedent on the application. If network
should behave as a binary classifier, than we can use one neuron with sigmoid 
activation fcuntion. If the aim is to create classify n categories, the layer should
contain n neurons followed by a soft-max fucntion. In the case of localization of a single
object, four neurons can predict two corners of a box suronding the target. The whole Architecture
can be seen on figure TODO.


\section{Training}
During the training of a network, its free parametes, weights and biases, are being changed to values,
which make the model perform well on the training dataset. The performance of a model is meassured by a 
loss function. In the case of regression, sum-of-squared errors can be used to compute the fit

$$R\(\theta\) = finish$$.

To evaluate classifier can be used cross-entropy:

$$R\(\theta\) = finish$$.

The loss of a model is typicaly reduced by stochastic gradient descent, in a case of neural network also
called backpropagation - the gradient is propagated back trough the network. \cite{hastie2009elements}

%%%%%%%%%%%%%%%%%%%%%%%%%%%%%%%%%%%%%%%%%%%%%%%%%%%%%%%%%%%%%%%%%%%%%%%%%%%%%%%%

\chapter{Implementation details}
The project was implemented in programing language Python \cite{Python}, version 3.7.
For using, creating and training neural network is used deep learning library PyTorch\cite{PyTorch}.
Code documentation is following numpy docstring guide \cite{NP}, and it is formatted by Black code 
formater \cite{Black}


\section{Project structure}
The project is divided into three main parts: classification, localization and segmentation.
Code share between them is stored in the root directory. For each part there files which create
models, datasets and may contain some additional funcionality, for example data data augmentation.
The complete structure can be seen on figure TODO.

\section{Examples}
For every part there is an example script which traines sample model on specified dataset. For
every type of model there is a script with example usage as well as samle data on which these 
models can be ran. All of the examples scripts can be found the the root directory.


\chapter{List of abbreviations}
To write.

\printindex

\appendix

\bibliographystyle{amsalpha}
\bibliography{kostema3}

\ctutemplate{specification.as.chapter}

\end{document}
