
%!TEX ROOT=ctutest.tex

\chapter{Goals and objectives}
The aim of this work is to create a collection of machine learning models, which
will be used in both, further research and image processing pipeline. The first
goal is to create a model which would be able to categorize different types of
ultrasound images, namely transversal, longitudinal and doppler ones. Further, 
we need to localize the common carotid artery (respectively internal carotid artery)
on the images of the first two named categories. As a last step, we need to
design a segmentation model which would segment three particular parts of
an artery with stenosis - vessel, lumen and plaque. Since all of these steps
in processing are interconnected and the outputs of one model are fed into a 
next one, all of the proposed methods need to achieve sufficient performance.

%%%%%%%%%%%%%%%%%%%%%%%%%%%%%%%%%%%%%%%%%%%%%%%%%%%%%%%%%%%%%%%%%%%%%%%%%%%%%%%%

\chapter{Background}

\section{Carotid Artery Stenosis}
To write.

\section{Medical ultrasound}
In general, medical ultrasound is a technique using sound waves in range etc.


%%%%%%%%%%%%%%%%%%%%%%%%%%%%%%%%%%%%%%%%%%%%%%%%%%%%%%%%%%%%%%%%%%%%%%%%%%%%%%%%

\chapter{Existing methods}

\section{Classification}
Image classification is one of the primary tasks in the field od image processing.
The neural network have achieved breakthrough in this field, namely ones using
convolutional layers. Later, as in many other domains, the deep learning has 
become state of the art in this field. One of the benchmark for this task is ImageNet 
Large Scale Visual Recognition Challenge \cite{russakovsky2015imagenet}, which has begun in 2010.
The task is to crate a model able to classify over 1.4 million images into one thousand 
categories. The size of the annotated dataset with the reduction of training time achieved by using GPU led
to deep architectures \cite{AlexNet}.


\subsection{Image classification in medicine}


\subsection{VGG}
Very deep convolutional neural networks for large-scale image recognition \cite{VGG} were 
introduced in 2014, and it achieved both, first and second class in the Classification tracks of 
ImageNet Challenge \cite{IN2014}.


\subsubsection{Architecture}
Six architectures were introduced in the experiments, each containing six blocks of convolutional
layers separated by max-pooling ones. In the convolutional layers were used filters with
with size 3x3 (in one experiment were used filters with size 1x1 at the end of three convolutional blocks).
With the 3x3 convolutional filters is used stride 1 with zero padding, so the spatial dimensionality
is preserved trough whole block. In the max-pooling layer, a receptor field with size 2x2 is used and with stride 1 it 
reduces the dimensionality to half. Finally, there are three fully-connected layers. First two with 4096 neurons and
the last one with 1000 neuron followed by sigmoid activation function. The two best performing models with 16, respectively 19
layers are described by Figure TODO.


%%%%%%%%%%%%%%%%%%%%%%%%%%%%%%%%%%%%%%%%%%%%%%%%%%%%%%%%%%%%%%%%%%%%%%%%%%%%%%%%


\subsection{ResNet}
The deep neural networks are generally harder to train \cite{HardDnn}. In 2015 was proposed ResNet\cite{ResNet}, 
a deep residual convolutional network. The depth of the networks was pushed even further, up to 152 layers. 


\subsubsection{Architecture}
To write.


%%%%%%%%%%%%%%%%%%%%%%%%%%%%%%%%%%%%%%%%%%%%%%%%%%%%%%%%%%%%%%%%%%%%%%%%%%%%%%%%


\section{Localization}
To write.

\subsection{Faster R-CNN}
To write.

\subsubsection{Architecture}
To write.

\subsubsection{Applications}
To write.


%%%%%%%%%%%%%%%%%%%%%%%%%%%%%%%%%%%%%%%%%%%%%%%%%%%%%%%%%%%%%%%%%%%%%%%%%%%%%%%%


\section{Segmentation}
To write.

\subsection{U-net}
To write.

\subsubsection{Architecture}
To write.

\subsubsection{Applications}
To write.

%%%%%%%%%%%%%%%%%%%%%%%%%%%%%%%%%%%%%%%%%%%%%%%%%%%%%%%%%%%%%%%%%%%%%%%%%%%%%%%%
